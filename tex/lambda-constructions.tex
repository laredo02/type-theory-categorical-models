\subsection{\centering Some Important Constructions}

Within the \lcalc \ there exist a number of conventional constructions that represent key ideas in programming (numbers, booleans, branching and lists among others) In the following we define a few of the most important ones. These constructions can be used to prove relevant statements concerning the expressiveness of the \lcalc.

We begin defining $ \mathbb{N} $ for no particular reason. The notion of natural numbers in the \lcalc \ follows from the Peano axioms:
\begin{definition} Peano Axioms, the set of natural numbers $\mathbb{N}$
\begin{itemize}
\item (Zero) \( 0 \in \NAT \)
\item (Successor) \( \forall n \in \NAT, S(n) \in \NAT \)
\item (Initial) \( \forall n \in \NAT, S(n) \neq 0 \) \label{def:peano-nat-initial}
\item (Injectivity) \( \forall n, m \in \NAT, S(n) = S(m) \Rightarrow n = m \)
\item (Induction) \( P(0) \land \forall ( P(n) \Rightarrow P(S(n))) \Rightarrow \forall n,  P (n) \)
\end{itemize}
The first four axioms can be summarized using a graph:
\begin{center}
  \begin{tikzpicture}[
    roundnode/.style={circle, draw=black!60, fill=gray!20, very thick, minimum size=7mm},
    emptynode/.style={circle, draw=black!0, fill=gray!20, very thick, minimum size=8mm}
    ]
    \node[roundnode]      (zero)                        {\textbf{0}};
    \node[roundnode]      (one)       [right=of zero]   {1};
    \node[roundnode]      (two)       [right=of one]    {2};
    \node[roundnode]      (three)     [right=of two]    {3};
    \node[emptynode]      (dots)      [right=of three]  {...};
    \node[roundnode]      (nth)       [right=of dots]   {n};

    \draw[->] (zero.east) -- node[midway, above] {s} (one.west);
    \draw[->] (one.east) -- node[midway, above] {s} (two.west);
    \draw[->] (two.east) -- node[midway, above] {s} (three.west);
    \draw[->] (three.east) -- node[midway, above] {s} (dots.west);
    \draw[->] (dots.east) -- node[midway, above] {s} (nth.west);
  \end{tikzpicture}
\end{center}
The last axiom is concerned with proving properties for all naturals given a precedent that satisfies the property and proof that if the property is satisfied for $n$ then it is satisfied for $n+1$, then the property is true of all naturals.
\end{definition}
\begin{example}
  Say for instance we mean to represent $ 3 $ using Peano notation, using the graph, we would come up with $ s(s(s(0))) $, since it is $3$ steps away from $0$.
\end{example}
\begin{definition} To add and multiply natural numbers one can just use following the sets of rules:
  \[
    n + m = 
    \begin{cases}
      n & \text{if } m = 0 \\
      s(n + m') & \text{if } m = s(m')
    \end{cases}
    \quad
    a \cdot b =
    \begin{cases}
      0 & \text{if } b = 0 \\
      a \cdot b' + a & \text{if } b = s(b')
    \end{cases}
  \]
\end{definition}
\begin{itemize}
\item \underline{$a + b$}: A recursive rule that performs the addition, and a base case to stop the recursion for n + 0, since there is no such number that has $ 0 $ as succesor.
\item \underline{$a \cdot b$}: Using addition we define multiplication in a similar way, adding $a$ as many times as the function $s$ is applied to $0$ in $b$.
\end{itemize}
\begin{definition} Representation of the set of all naturals in the realm of \lcalc, usually referred to as Church numerals:
  \label{def:church-naturals}
  \begin{align*}
    \texttt{0} &\equiv \la f.\la x. x \\
    \texttt{SUCC} &\equiv \la n. \la f. \la x . f ( n f x )
  \end{align*}
\end{definition}
\begin{example}
  The church numerals have been defined similarly to the peano axioms, that is, by bringing into existance an initial element and using the succesor function to define the rest of them. Let us apply \texttt{SUCC} a few times to illustrate:
  \begin{align*}
    \texttt{0} &\equiv \la f.\la x. x \\
    \texttt{1} &\equiv \la f.\la x. f x \\
    \texttt{2} &\equiv \la f.\la x. f f x \\
    \texttt{3} &\equiv \la f.\la x. f f f x \\
    \texttt{n} &\equiv \la f.\la x. f ^ n x
  \end{align*}
  where $ n $ denotes the number of times we append $ f $.
\end{example}
\begin{remark}
  Since \lterms \ are first-class citizens, and $f$ and $x$ are abstracted away, the reality is that numbers are a quite general construct that has many distinct applications. Recall Example \ref{ex:high-order-by-example} for instance.
\end{remark}
\begin{definition}
Now that we have a construction for the naturals, we can define some arithmetic operations:
\begin{align*}
  \texttt{ADD} &\equiv \la m.\la n. \la f. \la x. m f ( n f x) \\
  \texttt{MUL} &\equiv \la f.\la x. \la f. \la x. m (n f) x \\
  \texttt{ISZERO} &\equiv \lambda n. n\,(\lambda x.\texttt{FALSE})\,\texttt{TRUE}
\end{align*}
\end{definition}
\begin{example}
  \begin{center}
    \begin{align*}
      \texttt{ADD}\;\texttt{2}\;\texttt{3} &= (\la m.\la n.\la f.\la x. m\,f\,(n\,f\,x))\;(\la f.\la x. f\,f\,x)\;(\la f.\la x. f\,f\,f\,x) \\
                                           &\curly^* \la f.\la x.\; (\la f.\la x. f\,f\,x)\;f\;((\la f.\la x. f\,f\,f\,x)\;f\;x) \\
                                           &\curly^* \la f.\la x.\; (\la x. f\,f\,x)\;((\la x. f\,f\,f\,x)\;x) \\
                                           &\curly (\la f.\la x.\; f\,f\,f\,f\,f\,x) \\
                                           & \equiv \texttt{5} \quad \text{\magenta{Can I use alpha equiv o is it confusing?}}
    \end{align*}
  \end{center}
Multiplication works similarly to addition. As for \texttt{ISZERO}, the intuition behind it is that if the number is zero, then \(\lambda x.\, \texttt{FALSE}\) is dropped, and the result is \texttt{TRUE}; otherwise, it is applied at least once, which immediately yields \texttt{FALSE}.
\end{example}
\begin{remark}
  This proves that the \lcalc \ is capable of arithmetic, we leave them as a sidequest since we will not be extending or using them or in the rest of the text.
\end{remark}
The next constructions in line are the truth values of boolean logic, ``\textit{true}'' and ``\textit{false}'' for short.
\begin{definition} Boolean truth values:
  \begin{align*}
    \texttt{TRUE} &\equiv \lambda t.\lambda f. t \\
    \texttt{FALSE} &\equiv \lambda t.\lambda f. f
  \end{align*}
\end{definition}
\begin{remark}
  \label{rem:true-false}
  Note how $\TRUE \ M \ N \curly M $ and $ \FALSE \ M \ N \curly N $
\end{remark}
\begin{definition} Conditional, in other words \textit{``if statement''}.
  \begin{align*}
    \texttt{IF} &\equiv \lambda b.\lambda t.\lambda e. b\,t\,e
  \end{align*}
\end{definition}
The reality of the \IF \ is that it is just a combinator that happens to behave similarly to a branching statement whenever we pass a church encoded boolean as a first argument.
\begin{example} On the correct behabiour of $\IF$ i.e. check that it implements branching. $ M, N \in \La $ and $ \IF, \TRUE, \FALSE$ are as usual:
  \[
    \begin{aligned}
      & \IF \ \TRUE \ M \ N \\
      \equiv \ & (\la b.\la t.\la e.\, b\, t\, e) \ \TRUE \ M \ N  \\
      \curly \ & (\la t.\la e.\, \TRUE\ t\, e) \ M \ N  \\
      \curly \ & (\la e.\, \TRUE\ M\, e) \ N  \\
      \curly \ & \TRUE\ M\, N \curly^* M 
    \end{aligned}
    \quad
    \begin{aligned}
      & \IF \ \FALSE \ M \ N \\
      \curly^* \ & \FALSE\ M\, N \curly^* N
    \end{aligned}
  \]
  So the \IF \ just combines $b, t, e$ in such a way that whenever b is a church encoded boolean either $t$ or $e$ is dropped depending on the truth value of $b$. See Remark \ref{rem:true-false}. From this example we extract that $ \la x . x $ is equivalent to \IF \ it terms of behaviour, and thus, we could use $ \IF \equiv \la x . x $ interchangeably.
\end{example}
\begin{remark}
  Using the \IF \ we can implement an universal set of logic gates:
  \begin{align*}
    \NOT &\equiv \IF \, b \, \FALSE \, \TRUE \\
    \AND &\equiv \IF \, a \, (\IF \, b \, \TRUE \, \FALSE) \, \FALSE
  \end{align*}
  This allows us to simulate a nand gate, known to be universal \orange{\ref{} universality, conjunctive normal form}. Thus, the lambda calculus is at least equivalent to propositional logic \orange{\ref{} Something on turing completeness}.
\end{remark}
On the same track, another must-have for programmers is lists. Sice they are a bit more involved, we require tuples to define them. Note that the \IF \ is a tuple that contains two elements, we query the first element using a \TRUE \ i.e. the \textit{``then''} branch, and the sencond using a \FALSE \ i.e. the \textit{``else''} branch. We will be using this alternative definition for convenience:
\begin{definition} \( \PAIR \equiv \la x.\la y.\la f. f\,x\,y \)
\end{definition}
\begin{remark}
  This is more convenient since $ \PAIR \, M \, N \curly^* \la f . f \, M \, N  $, which allows us to query the first element using $ (\la f . f \, M \, N) \, \TRUE $ and the second by $ (\la f . f \, M \, N) \, \FALSE $, this motivates the definition below:
\end{remark}
\begin{definition}
\begin{align*}
  \FST &\equiv \la p. p\,(\la x.\la y. x) \\
  \SND &\equiv \la p. p\,(\la x.\la y. y)
\end{align*}
\end{definition}
\begin{proposition} A key property of $ \PAIR $'s is:
  \[\PAIR \, A \, B \equiv_\beta \PAIR \, (\FST \, A \, B) \, (\SND \, A \, B) \]
\end{proposition}
Using pairs, we now can define linear lists, we can achieve this using either left linear trees or right linear trees:
\begin{center}
  \[
    \begin{aligned}
      \begin{forest}
        [\rotatebox{60}{\texttt{\ldots}}
        [\PAIR
        [\PAIR
        [\NIL]
        [\texttt{ELEM}]
        ]
        [\texttt{ELEM}]
        ]
        [\texttt{ELEM}]
        ]
      \end{forest}
    \end{aligned}
    \hspace{1cm}
    \textit{or}
    \hspace{1cm}
    \begin{aligned}
      \begin{forest}
        [\rotatebox{-60}{\texttt{\ldots}}
        [\texttt{ELEM}]
        [\PAIR
        [\texttt{ELEM}]
        [\PAIR
        [\texttt{ELEM}]
        [\NIL]
        ]]]
      \end{forest}
    \end{aligned}
  \]
\end{center}
\begin{definition} We define \NIL \ in the same way we defined \texttt{0} and \FALSE \ before: $ \NIL \equiv \la f.\la z. z $; \NIL \ represents the terminating element of the inductive definition for lists. To build lists, we cons lists together using:
  \[ \texttt{CONS} \equiv \la h.\la t.\la f.\la z. f\,h\,(t\,f\,z) \]
that binds a new element $h$ to to an pre-existing list or to \NIL.
\end{definition}
\begin{note}
  This notion of inductively defined lists is lays the foundations of the \textbf{LISP} family of programming languages, hence the name (LISt Processing).
\end{note}
\begin{definition} Relevant list operators:
\begin{align*}
  \texttt{ISNIL} &\equiv \la l. l\,(\la h.\la t. \FALSE)\,\TRUE \\
  \texttt{HEAD} &\equiv \la l. l\,(\la h.\la t. h)\,\text{undef}
\end{align*}  
\end{definition}


\newpage
\magenta{Turing completeness of the \lcalc}
\begin{align*}
  \texttt{CONS} &\equiv \la h.\la t.\la f.\la z. f\,h\,(t\,f\,z) \\
  \texttt{IS\_NIL} &\equiv \la l. l\,(\la h.\la t. \texttt{FALSE})\,\texttt{TRUE} \\
  \texttt{HEAD} &\equiv \la l. l\,(\la h.\la t. h)\,\text{undef} \\
  \texttt{TAIL} &\equiv \la l.\, \texttt{FIRST}\,(l\,(\la p.\la h.\, \texttt{PAIR}\,(\texttt{SECOND}\,p)\,(\texttt{CONS}\,h\,(\texttt{SECOND}\,p)))\,(\texttt{PAIR}\,\texttt{NIL}\,\texttt{NIL}))
\end{align*}
\textbf{Recursion:}
\begin{align*}
  \texttt{Y} &\equiv \lambda f.(\lambda x. f\,(x\,x))\,(\lambda x. f\,(x\,x))
\end{align*}
