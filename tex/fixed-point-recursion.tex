\subsection{\centering Fixed Point Combinators and Recursion}
\label{sec:recursion-fixed-points}

The Church Rosser Property labels the untyped \lcalc \ as a confluent system, therefore we know that whenever a term has a normal form it must be unique. \orange{what we did cover} We still have not given an answer to whether all \lterms \ have a normal form.

The most straight forward way to test this, is to check whether there exists some term that reduces to itself, effectively creating a loop in the $\beta$-reduction chain so to speak. In more formal terms this means there exists some $M$ such that \( M \curly^n M \) with $ n > 0 $. It turns out \( \Omega \equiv (\la x . \, x \, x) (\la x . \, x \, x) \) fullfills this requirement. Let us inspect what goes on while $\beta$-reducing this term:
\[
  \Omega \equiv (\la x . \, x \, x) (\la x . \, x \, x) \curly (\la x . \, x \, x) (\la x . \, x \, x) \curly^* (\la x . \, x \, x) (\la x . \, x \, x) \equiv \Omega
\]
This so called $\Omega$-combinator, proves that not all terms have a normal form, and shines a beam of light onto how recursion can be achieved in the \lcalc.

There is one caveat to this, that classically, recursion is achieved through naming, take for instance this definition for the $\mathrm{gdc}$:
\begin{align*}
  \underline{\mathrm{gcd}}(a,b) &= 
                      \begin{cases}
                        a & b = 0 \\
                        \underline{\mathrm{gcd}}(b,\, a \bmod b) & b \neq 0
                      \end{cases}
\end{align*}
Here, when $b \neq 0$ the name gcd is used to refer to the same definition again. In the \lcalc \ such self-referencing definitions are not permited since the language does not allow names. To bypass this setback the notion of fixed point is of particular interest.
\begin{definition}
  Let \( f : X \to X \) be a function. An element \( x \in X \) is called a fixed point of \( f \) if \( f(x) = x. \)
\end{definition}
\begin{example} Some simple functions that illustrate what fixed points are:
  \[
    \begin{tabular}{|c|c|}
      \hline
      \textit{Function} & \textit{Fixed Point(s)} \\
      \hline
      $f(x) = c$ & $x = c$ \\
      \hline
      $g(x) = x$ & $\{ x \mid x \in \mathbb{R} \}$ \\
      \hline
      $h(x) = x^2$ & $ x = 0,\, x = 1 $ \\
      \hline
    \end{tabular}
  \]
\end{example}
This harmless idea satisfies an interesting property, namely that $ x = f(x) = f(f(x)) = f^n(x) $ effectively creating a chain where every element $ f^{n+1}(x) = f(f^{n}(x))$, thus attaining self-reference without naming. This idempotent behaviour under application is important, since it showcases the ability of fixed points to induce recursion. Our quest now, is to find a term that simulates this behavior. To do this, let us translate the definition for fixed point into the $\lambda$-world.
\begin{definition}
  Given a lambda term \( L \in \La \), a term \( M \in \La \) is called a fixed point of \( L \) if \( M =_\beta L M \).
\end{definition}
No surprises here, the definition is close to a one-to-one mapping of its functional counterpart seen previously. Interestingly enough, it turns out that every lambda term has a fixed point, as stated in the theorem proven below:
\begin{theorem}
  For every \( L \in \La \), there exists \( M \in \La \) such that $ M $ is a fixed point for $ M $ i.e. $ M =_\beta L M $. Referred to as the Fixed Point Theorem in the literature.
\end{theorem}
\begin{proof} Given $L \in \La$, define $M$ as below:
  \[
    M := (\la x . \, L \, (x \, x))(\la x . \, L \, (x \, x))
  \]
  $M$ is a reducible expression, redex for short:
  \begin{align*}
    M 
    &\equiv (\la x . \, L \, (x \, x))(\la x . \, L \, (x \, x)) \\
    &\curly L \, ((\la x . \, L \, (x \, x))(\la x . \, L\, (x \, x))) \\
    &\equiv L \, M.
  \end{align*}
  Hence, $ M =_\beta L M $, as required.
\end{proof}
\begin{remark}
  See how $ L M $ is the application of $ M $ to $ L $ and that using \textsc{AppR} we can keep $\beta$-reducing $ M $ the same way we did for $ x = f(x) = f(f(x))\dots $ only this time using lambda terms:
  \[
    M \curly L \, M \curly L \, L \, M \curly^* L^n M.
  \]
\end{remark}
Now that nameless self-reference has been proven to exist within the the \lcalc, we take a step forward and introduce a few of the most important fixed point combinators and explore their different natures.
\begin{definition} $Y$ Combinator. \orange{The term that given some other term returns its fixed point.}
\[
  Y \triangleq \la f.\, (\la x.\, f\, (x\, x))\, (\la x.\, f\, (x\, x))
\]
\end{definition}
\begin{note}
  \( Y \equiv \la L . M  \) \magenta{M defined previously. Improve this, find some other way to put it? note it is an improved $\Omega$}
\end{note}
\begin{remark}
  Even though the $Y$ combinator is of theoretical interest since it allows us to find recursion within the \lcalc, it is of little practical use since it always runs into non-termination e.g.
  \[
    \begin{aligned}
      Y \, F &\equiv (\la f. \, (\la x . \, f \, (x \, x))(\la x . \, f \, (x \, x))) F \\
      &\curly (\la x . \, F \, (x \, x))(\la x . \, F \, (x \, x)) \\
             &\curly F((\la x . \, F \, (x \, x))(\la x . \, F \, (x \, x))) \\
             &\curly F(F((\la x . \, F \, (x \, x))(\la x . \, F \, (x \, x)))) \\
             &\curly \cdots
    \end{aligned}
  \]
  This is due to its lazy nature, this means that the argument $F$ is not evaluated until $Y$ is done unravelling i.e. never.
\end{remark}
To work around this issue, several other combinators exist, one of the most well-known, the $Z$ combinator is defined next:
\begin{definition} $Z$ combinator
\[
  Z \triangleq \la f.\, (\la x.\, f\, (\la v.\, x\, x\, v))\, (\la x.\, f\, (\la v.\, x\, x\, v))
\]
\end{definition}
\begin{example}
  \[
    \begin{aligned}
      Z \, F &\equiv (\la f.\, (\la x.\, f\, (\la v.\, x\, x\, v))\, (\la x.\, f\, (\la v.\, x\, x\, v)))\, F \\
             &\curly (\la x.\, F\, (\la v.\, x\, x\, v))\, (\la x.\, F\, (\la v.\, x\, x\, v)) \\
             &\curly F\, (\la v.\, (\la x.\, F\, (\la v.\, x\, x\, v))\, (\la x.\, F\, (\la v.\, x\, x\, v))\, v)\, \\
             &\equiv F\, (\la v.\, Z \, F \, v)\, \\
    \end{aligned}
  \]
\end{example}











