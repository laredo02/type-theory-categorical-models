
\usepackage[a4paper, margin=4cm]{geometry}
\usepackage[english]{babel}
\usepackage{amsmath}
\usepackage{amssymb}
\usepackage{amsthm}
\usepackage{syntax}
\usepackage{tikz}
\usepackage{xcolor}
\usepackage{forest}
\usepackage{mathtools}
\usepackage{tikz-cd}
\usepackage{mathrsfs}
\usepackage{stmaryrd}
\usepackage{enumitem}
\usepackage{listings}
\usepackage{xcolor}
\usepackage{stmaryrd}
\usepackage{bussproofs}
\usepackage{amsmath,amssymb}
\usepackage{mathtools}
\usepackage{stmaryrd}
\usepackage{mathpartir}
\usepackage{tikz}
\usetikzlibrary{positioning, arrows.meta}
\definecolor{refcolor}{RGB}{0, 0, 0}
\usepackage[
  colorlinks=true,
  linkcolor=refcolor,
  citecolor=refcolor,
  urlcolor=refcolor,
  pdfborder={0 0 0}
]{hyperref}
\usepackage{xcolor}
\usepackage{etoolbox}
\definecolor{keywordcolor}{rgb}{0.0,0.2,0.6}
\definecolor{commentcolor}{rgb}{0.3,0.5,0.3}
\definecolor{stringcolor}{rgb}{0.5,0.1,0.1}
\definecolor{bgcolor}{rgb}{0.95,0.95,0.95}
\lstdefinelanguage{Haskell}{
  morekeywords={
    case,class,data,default,deriving,do,else,if,import,in,infix,infixl,
    infixr,instance,let,module,newtype,of,then,type,where,forall
  },
  sensitive=true,
  morecomment=[l]--,
  morecomment=[n]{\{-}{-\}},
  morestring=[b]",
}

\lstdefinestyle{haskellstyle}{
  language=Haskell,
  basicstyle=\ttfamily\small,
  keywordstyle=\color{keywordcolor}\bfseries,
  commentstyle=\color{commentcolor}\itshape,
  stringstyle=\color{stringcolor},
  backgroundcolor=\color{bgcolor},
  showstringspaces=false,
  numbers=left,
  numberstyle=\tiny,
  numbersep=5pt,
  frame=single,
  breaklines=true,
  captionpos=b
}

\newcommand{\la}{\lambda}
\newcommand{\La}{\mathbf{\Lambda}}
\newcommand{\Vset}{\mathbf{V}}
\newcommand{\NAT}{\mathbb{N}}
\newcommand{\IF}{\texttt{IF}}
\newcommand{\SUCC}{\texttt{SUCC}}
\newcommand{\TRUE}{\texttt{TRUE}}
\newcommand{\FALSE}{\texttt{FALSE}}
\newcommand{\FST}{\texttt{FST}}
\newcommand{\SND}{\texttt{SND}}
\newcommand{\PAIR}{\texttt{PAIR}}
\newcommand{\NIL}{\texttt{NIL}}
\newcommand{\NOT}{\texttt{NOT}}
\newcommand{\AND}{\texttt{AND}}
\newcommand{\OR}{\texttt{OR}}
\newcommand{\cat}[1]{\mathbf{#1}}
\newcommand{\Hom}{\mathrm{Hom}}
\newcommand{\id}{\mathrm{id}}
\newcommand{\Ob}{\mathrm{Ob}}
\newcommand{\Mor}{\mathrm{Mor}}
\newcommand{\Set}{\cat{Set}}
\newcommand{\Top}{\cat{Top}}
\newcommand{\Grp}{\cat{Grp}}
\newcommand{\Fun}{\mathrm{Fun}}
\newcommand{\op}{^{\mathrm{op}}}
\newcommand{\lCalc}{$\la$-Calculus}
\newcommand{\lcalc}{$\la$-calculus}
\newcommand{\bred}{$\beta$-reduction}
\newcommand{\lterm}{$\la$-term}
\newcommand{\lterms}{$\la$-terms}
\newcommand{\lexpr}{$\la$-expression}
\newcommand{\aequiv}{$\alpha$-equivalence}
\newcommand{\aequivlt}{$\alpha$-equivalent}
\newcommand{\bequiv}{$\beta$-equivalence}
\newcommand{\functionfont}[1]{\mathbf{#1}}
\newcommand{\curly}{\mathrel{\leadsto}_\beta}
\newcommand{\Sub}[1]{\functionfont{Sub}(#1)}
\newcommand{\Subb}{\functionfont{Sub}}
\newcommand{\FV}[1]{\functionfont{FV}(#1)}
\newcommand{\FVV}{\functionfont{FV}}
\newcommand{\orange}[1]{\textcolor{orange}{#1}}
\newcommand{\magenta}[1]{\textcolor{magenta}{#1}}
\newcommand{\blue}[1]{\textcolor{blue}{#1}}
\newcommand{\green}[1]{\textcolor{green}{#1}}
\newcommand{\myref}[1]{\textcolor{refcolor}{\hyperref[#1]{\ref*{#1}}}}
\newcommand{\subst}[2]{[#1 := #2]}
\newcommand{\step}{\curly}
\newcommand{\etaequiv}{$\eta$-equivalence}
\newcommand{\lam}[2]{\lambda #1.\,#2}
\newcommand{\app}[2]{#1.\,#2}
\newcommand{\abs}[2]{\la #1.\,#2}
\newcommand{\curlysym}{\mathrel{\leftrightarrow}_\beta}
\renewcommand{\arraystretch}{1.2}
\setlength{\tabcolsep}{12pt}

\theoremstyle{definition}
\newtheorem{theorem}{Theorem}[section]
\newtheorem{lemma}[theorem]{Lemma}
\newtheorem{corollary}[theorem]{Corollary}
\newtheorem{definition}{Definition}[section]
\newtheorem{example}{Example}[section]
\newtheorem{remark}{Remark}
\newtheorem{proposition}[theorem]{Proposition}
\newtheorem{note}{Note}

\bibliographystyle{alpha}

\title{Type Theory and Categorical Models: A Unified Approach}
\author{Miguel Laredo}



