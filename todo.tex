% Created 2025-04-01 mar 00:47
% Intended LaTeX compiler: pdflatex
\documentclass[11pt]{article}
\usepackage[utf8]{inputenc}
\usepackage[T1]{fontenc}
\usepackage{graphicx}
\usepackage{longtable}
\usepackage{wrapfig}
\usepackage{rotating}
\usepackage[normalem]{ulem}
\usepackage{amsmath}
\usepackage{amssymb}
\usepackage{capt-of}
\usepackage{hyperref}
\author{Miguel Laredo}
\date{\today}
\title{Type Theory and Categorical Semantics: A Unified Approach}
\hypersetup{
 pdfauthor={Miguel Laredo},
 pdftitle={Type Theory and Categorical Semantics: A Unified Approach},
 pdfkeywords={},
 pdfsubject={},
 pdfcreator={Emacs 29.3 (Org mode 9.6.15)}, 
 pdflang={English}}
\begin{document}

\maketitle
\tableofcontents


\section{Chapter 1: Lambda Calculus \& Type Systems}
\label{sec:org9ae00cd}
\subsection{From Untyped to Dependent Types}
\label{sec:org8857005}
\subsubsection{1. Untyped Lambda Calculus}
\label{sec:org3ae57e4}
\begin{itemize}
\item Syntax: \(\lambda\), application, variables
\item Reduction: \(\alpha\), \(\beta\), \(\eta\)
\item Church encodings (Booleans, Naturals)
\item Fixed-point combinators (Y) and recursion
\item Limitations: Non-termination, lack of safety
\end{itemize}

\subsubsection{2. Simply Typed Lambda Calculus (STLC)}
\label{sec:org4954340}
\begin{itemize}
\item Types: base, functions (→), products/sums (×, +)
\item Typing judgments (Γ ⊢ M : τ)
\item Normalization (strong vs. weak)
\item Curry-Howard preview: STLC ≃ Propositional Logic
\end{itemize}

\subsubsection{3. Parametric Polymorphism (System F)}
\label{sec:org33a71b2}
\begin{itemize}
\item Type abstraction/application (Λα. M, M [τ])
\item Theorems for Free (parametricity)
\item Limitations: No value-dependence
\end{itemize}

\subsubsection{4. Dependent Types (λΠ, Martin-Löf)}
\label{sec:org99b8870}
\begin{itemize}
\item Π-types (∀), Σ-types (∃), equality types (Id)
\item Inductive types (Nats, Lists)
\item Example: Vectors (Vec A n), certified sorting
\end{itemize}

\section{Chapter 2: Logic \& the Curry-Howard Isomorphism}
\label{sec:orga258183}
\subsection{From Proofs to Programs}
\label{sec:org88ce455}
\subsubsection{1. Propositional Logic}
\label{sec:org496ea13}
\begin{itemize}
\item Natural deduction: →, ∧, ∨, ⊥ rules
\item Intuitionistic vs. classical (excluded middle)
\item BHK interpretation (proofs as programs)
\end{itemize}

\subsubsection{2. Curry-Howard for STLC}
\label{sec:org7e9739b}
\begin{itemize}
\item Propositions-as-Types (A → B ≈ function type)
\item Proofs-as-Terms (Modus Ponens ≈ application)
\item Examples:  
\begin{itemize}
\item A → (B → A) ≈ λa. λb. a (K combinator)
\item ¬A ≈ A → ⊥
\end{itemize}
\end{itemize}

\subsubsection{3. Predicate Logic \& Dependent Types}
\label{sec:org98eacb9}
\begin{itemize}
\item Universal quantifier (∀ ≈ Π-type)
\item Existential quantifier (∃ ≈ Σ-type)
\item Example: ∀n ∈ ℕ. ∃m ∈ ℕ. m > n ≈ dependent function + pair
\end{itemize}

\subsubsection{4. Proof Theory \& Computation}
\label{sec:orgb856af5}
\begin{itemize}
\item Cut elimination ≈ β-reduction
\item Normalization ≈ program termination
\end{itemize}

\section{Chapter 3: Categorical Semantics}
\label{sec:org2cccdfa}
\subsection{Types as Objects, Programs as Morphisms}
\label{sec:orgdcacef9}
\subsubsection{1. Categories of Types}
\label{sec:orgb2cfa7f}
\begin{itemize}
\item Cartesian Closed Categories (CCC) for STLC:  
\begin{itemize}
\item Products (A × B) ≈ conjunction (A ∧ B)
\item Exponentials (B\textsuperscript{A}) ≈ implication (A → B)
\end{itemize}
\item Functorial semantics of System F
\end{itemize}

\subsubsection{2. Dependent Types \& LCCCs}
\label{sec:org8f40ca2}
\begin{itemize}
\item Locally Cartesian Closed Categories (LCCC):  
\begin{itemize}
\item Π-types ≈ right adjoints to substitution
\item Σ-types ≈ left adjoints
\end{itemize}
\item Fibrations for predicate logic
\end{itemize}

\subsubsection{3. Parametricity as Naturality}
\label{sec:orgda6cabd}
\begin{itemize}
\item Reynolds’ abstraction theorem ≈ dinaturality
\item Polymorphic functions as natural transformations
\end{itemize}

\subsubsection{4. Advanced Topics (Optional) [Maybe]}
\label{sec:orga5cb5b8}
\begin{itemize}
\item Monads/comonads for effects
\item Homotopy Type Theory (univalence)
\end{itemize}

\section{Appendices}
\label{sec:orgc4545e1}
\subsection{C. Papers: Church, Reynolds, Martin-Löf, Hindley-Milner}
\label{sec:org86fe2e7}
\end{document}
